Developing enterprise software often requires composing several
libraries together with a large body of in-house code. Especially in
object-oriented languages like Java, these libraries might add
hundreds of classes, along with thousands of methods, constants, and
functions to the environment, which adds a great deal of complexity
for the developer to manage. Modern programming conventions favor
composing many ``small" objects together to create larger structures,
or to delegate functionality or resource acquisition to other modules
in a program. While useful, these techniques steepen the learning
curve. A developer typically knows the approximate structure of the
desired expression. However, often the first attempt at writing that
code results in an ill-typed code fragment.

We describe an algorithm and a tool called \ourTool that automatically
repairs code expressions based on the provided almost-correct code. At
the core of our algorithm is a graph construction that expresses the
relationships between the language's types and methods. Such approach
allows us to synthesize and repair expressions that are biased towards
a given criteria by setting the edge weights appropriately. We
implemented our algorithm as part of an IntelliJ IDEA plugin that
proposes corrections for ill-typed expressions in the Java language.
\ruzica{here has to go something about the running times}
