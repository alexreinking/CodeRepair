In our previous work we developed a tool called InSynth
\cite{GveroETAL13CompleteCompletionTypesWeights} that automatically
synthesizes code snippets based on the given type constraints. InSynth
considers all user-defined declarations, together with any imported API
calls, when performing the synthesis. In principle, a na\" ive algorithm for code repair based on InSynth
could to solve the repair problem. The algorithm would first extend the
initial environment with all type declarations that could be derived from the
given ill-typed expression. Using the new environment we would run the
InSynth algorithm. Because the InSynth algorithm is complete,
eventually it will generate expressions (if they exist) following the
structure of the given backbone expression. However, this may not be
practical because the type inhabitation problem is a PSPACE-complete
problem and the ranks of resulting expressions is determined using heuristic
that ignores given ill-typed expression. The aim of our approach is to 
design a better and more efficient algorithm that is well-suited for
repairing ill-typed expressions.


The input to our algorithm is an ill-typed, backbone expression. We propose
two different algorithms. The first algorithm generates one well-typed expression
that strictly follows the structure of the input expression. It decomposes
the problem into finding connections between individual symbols in the
backbone expression. The connections are build using a given set of
repair declarations. Each connection represents a partial expression.
We find the smallest such expressions using a weights mechanism.
When we find all the partial expressions, we combine them following 
the original structure of the backbone expression. The algorithm 
simultaneously fixes all type errors in a given expression.
The second algorithm searches for the best solutions, not only
following the initial backbone expression, but also creating new ones
that are mutations of the original one. This might lead us to a
set of more interesting solutions in the case when the developer does not
provide us with the best initial backbone expression. Additionally, this approach 
allows completeness in sense that we are able to generate all 
expressions with the given type, similarly as in InSynth, but now ranked 
by similarity to the given backbone expression. The algorithm also 
implements A* search that steer us towards the most desirable
solutions.



The contributions of this paper are:
\begin{itemize}
	\item We formulate the problem of repairing ill-typed expressions.
	As input to the problem we take a backbone expression and 
	the set of repair declarations. We introduce weak long normal 
	form that allows a systematic search and a construction of the well-typed expressions.
	We identify special symbols used to extend the expressiveness of the input.
	\item We propose a novel repair calculus that specifies the rules that
	we use to derive a well-typed term from a backbone expression.
	The calculus describes how to fix a term using declarations with multiple arguments.
	\item We propose an algorithm that finds the best well-typed expression based on 
	weights system introduced in \cite{GveroETAL13CompleteCompletionTypesWeights}.
    We show that the algorithm is polynomial given a certain class of ill-typed expressions.
    \item We present a sound and complete algorithm that takes the ill-typed expression and 
    finds a set of best solutions. The algorithm is based on A* search.
\end{itemize}


