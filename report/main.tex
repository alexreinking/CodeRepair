%
% LaTeX template for prepartion of submissions to PLDI'15
%
% Requires sigplanconf style file provided on PLDI'15 web site
%
\documentclass[pldi]{sigplanconf}

%
% the following standard packages may be helpful, but are not required
%
\usepackage{SIunits}            % typset units correctly
\usepackage{courier}            % standard fixed width font
\usepackage[scaled]{helvet} % see www.ctan.org/get/macros/latex/required/psnfss/psnfss2e.pdf
\usepackage{url}                  % format URLs
\usepackage{listings}          % format code
\usepackage{enumitem}      % adjust spacing in enums
\usepackage[colorlinks=true,allcolors=blue,breaklinks,draft=false]{hyperref}   % hyperlinks, including DOIs and URLs in bibliography
% known bug: http://tex.stackexchange.com/questions/1522/pdfendlink-ended-up-in-different-nesting-level-than-pdfstartlink
\newcommand{\doi}[1]{doi:~\href{http://dx.doi.org/#1}{\Hurl{#1}}}   % print a hyperlinked DOI



\begin{document}

%
% any author declaration will be ignored  when using 'plid' option (for double blind review)
%

\title{Instructions for Submission to PLDI'15}

\maketitle
\begin{abstract}
  This document is intended to serve as a sample for submissions to PLDI'15,
  the 36th Annual ACM SIGPLAN Conference on Programming Language
  Design and Implementation.  We provide some guidelines
  that authors should follow when submitting papers to the conference.
\end{abstract}

\section{Introduction}

This document provides instructions for submitting papers to PLDI'15.
In an effort to respect the efforts of reviewers and in the interest
of fairness to all prospective authors, we request that all
submissions to PLDI'15 follow the formatting and submission rules
detailed below. \textbf{Submissions that violate these instructions
  may not be reviewed}, at the discretion of the
\href{mailto:steve.blackburn@anu.edu.au?subject=[PLDI'15]}{program
  chair}, in order to maintain a review process that is fair to all
potential authors.

An example submission (formatted using the PLDI'15 submission format)
that contains the submission and formatting guidelines can be
downloaded
\href{http://conf.researchr.org/getImage/pldi2015/orig/pldi15-template.pdf}{here}. The
content of this document mirrors that of the submission instructions
that appear on
\href{http://conf.researchr.org/track/pldi2015/pldi2015-papers#Submission-Instructions}{this website},
where the paper submission site will be linked online about four weeks
prior to the paper submission deadline.

\paragraph{Paper evaluation objectives}
The committee will make every effort to judge each submitted paper on
its own merits. There will be no target acceptance rate.  We expect to
accept a wide range of papers with appropriate expectations for
evaluation --- while papers that build on significant past work with
strong evaluations are valuable, papers that open new areas with less
rigorous evaluation are equally welcome and especially
encouraged. \textbf{In either case, what is important is that the
  paper's claims are no stronger than what is supported by the
  evaluation.}  Given the wide range of topics covered by PLDI, every
effort will be made to find expert reviewers.

All questions regarding paper formatting and submission should be
directed to the
\href{mailto:steve.blackburn@anu.edu.au?subject=[PLDI'15]}{program
  chair}.

\paragraph{Highlights From This Document}
\begin{itemize}[noitemsep]
\item Paper must be submitted in printable PDF format.
\item Text must be in a minimum \textbf{10\,pt font} (\emph{not} 9\,pt).
\item Papers at most \textbf{11 pages}, not including references. 
\item References in \textbf{author-year} style (as used by TOPLAS).
\item There is no page limit for references. 
\item Each reference must specify \textbf{all authors} (no \emph{et al.}). 
\item The reviewing process will include two phases.
  \begin{itemize}
  \item All papers will receive three reviews in the first phase.
  \item Those papers identified as most worthy of further consideration will proceed to the second phase and receive an additional two reviews.
  \item Authors of papers eliminated in the first phase will be notified promptly.
  \item Each phase will have an author response period.
  \item Authors \emph{will not} be required, nor given the opportunity, to revise their submissions after the initial paper deadline.
  \end{itemize}
\item Authors of \emph{all} accepted papers will be required to give a
  lightning presentation (about 90\,s) in addition to
  the regular conference talk.
\item Proceedings may appear in the ACM digital library as early as as early as May 30, 2015.
\end{itemize} 



\section{Paper Preparation Instructions}

\subsection{Paper Formatting}

Papers must be submitted in printable PDF format and should contain a
\textbf{maximum of 11 pages} of single-spaced two-column text,
\textbf{not including references}.  In-text citations must follow the
\textbf{ACM / TOPLAS author-year style (`[Smith 1990]')}, as in this
document, \textbf{not the numerical style (`[1]') formerly used by
  PLDI}.  The rationale for this change is that it improves
readability, and space is less of a concern for published proceedings
today than it once was.  You may include any number of pages for
references.  If you are using
\LaTeX~\cite{Lamport:94} to typeset your paper, then we suggest that
you use this updated version of the SIGPLAN 
\LaTeX~\href{http://conf.researchr.org/getImage/pldi2015/orig/sigplanconf.cls}{class
  file} and
\href{http://conf.researchr.org/getImage/pldi2015/orig/pldi15-template.tex}{template}.
(\href{http://conf.researchr.org/getImage/pldi2015/orig/pldi15-template.pdf}{This
  document} was prepared with that template.)  If you are using
Microsoft Word, you may wish to use
\href{http://conf.researchr.org/getImage/pldi2015/orig/pldi15-word-template.dot}{this}
template.  \textbf{Whichever tool you use, please ensure you adhere to the
guidelines given in Table~\ref{table:formatting}.}  The conference
submission website will use the
\href{https://www.usenix.org/legacy/event/wowcs08/tech/full_papers/voelker/voelker.pdf}{banal}
format checker to \emph{advise} on formatting compliance.

\begin{table}[t!]
  \centering
{  \sffamily\small % tabular data either 10pt times, or 9pt helvetica
  \begin{tabular}{rl}
    \textbf{Field} & \textbf{Value}\\
    \hline
    File format & PDF \\
    Page limit & 11 pages, \textbf{excluding references}\\
    Paper size & US Letter 8.5in $\times$ 11in\\
    Top margin & 1in\\
    Bottom margin & 1in\\
    Left margin & 0.75in\\
    Right margin & 0.75in\\
    Body & 2-column, single-spaced\\
    Column separation & 0.25in\\
    Body font & 10pt\\
    Tabular font & 10pt Times (or 9pt Helvetica)\\
    Abstract font & 10pt\\
    Section heading  & 12pt, bold\\
    Subsection heading  & 10pt, bold\\
    Caption font & 10pt\\
    In-text citation & Author-year (`[Smith 1990]' \emph{\textbf{not}} `[3]') \\
    References & 10pt, no page limit, list all authors' names\\
  \end{tabular}
}
  \caption{Formatting guidelines for submission. }
  \label{table:formatting}
\end{table}

For the \emph{convenience of reviewers}, all submissions will be
automatically watermarked by the paper submission site
(\href{http://conf.researchr.org/getImage/pldi2015/orig/pldi15-template-wm.pdf}{example
  here}), so please \textbf{avoid adding any text in the margins or
  headers} such as footers, headers, or banners, 
other than the centered page number provided by the template (which is
only for the author's convenience, since the watermarking adds page
numbers).  The watermarking process strips the submitted pdf of all
metadata (thereby removing information which might otherwise identify authors).

\subsection{Content}

\paragraph{Double Blind.}  Reviewing will be double blind; therefore,
please do not include any author names on any submitted documents
except in the space provided on the online submission form.  Please
take time to read the
\href{http://conf.researchr.org/track/pldi2015/pldi2015-papers#FAQ-on-Double-Blind-Reviewing}{PLDI
  FAQ on Double Blind Reviewing}, which gives a more comprehensive and
authoritative account than this short paragraph.  If you are improving
upon your prior work, refer to your prior work in the third person and
include a full citation for the work in the bibliography. For example,
if you happened to be Collins and McCarthy, building on \emph{your
  own} prior work, you might say something like: ``While prior work
\cite{Backus:60,Collins:60,McCarthy:60} did X, Y, and Z, this paper
additionally does W, and is therefore much better.''  \textbf{Do NOT omit or
anonymize references for blind review}.

\paragraph{Figures and Tables.} Ensure that the figures and tables
are legible.  Please also ensure that you refer to your figures in the main
text.  Many reviewers print the papers in gray-scale. Therefore, if you use
colors for your figures, ensure that the different colors are highly
distinguishable in gray-scale.

\paragraph{References.}  There is no length limit for references.
\textbf{Each reference must explicitly list all authors of the paper.}
  Papers not meeting this requirement will be rejected. Authors of
NSF proposals should be familiar with this requirement. Knowing all
authors of related work will help find the best reviewers. Since there
is no length limit for the number of pages used for references, there
is no need to save space here.

\section{Paper Submission Instructions}

\subsection{Declaring Authors}

Enter all authors of the paper into the online paper submission tool
upfront. Addition/removal of authors once the paper is accepted will
have to be approved by the \href{mailto:steve.blackburn@anu.edu.au?subject=[PLDI'15]}{program chair}, since it potentially
undermines the goal of eliminating conflicts for reviewer assignment.

\subsection{Supplementary Material}

The paper submission website will allow authors to upload an
additional document containing material that supplements the paper
(such as an extended proof or extensive results).  However, it is
essential that authors understand that: a) reviewers are \textbf{not
  obliged} to read the supplement, and b) the supplement must
  be \textbf{fully anonymized.}

\subsection{Areas and Topics}

PLDI is a broad conference.  It is not limited to the design and
implementation of programming languages, but reflects the diversity of
SIGPLAN.  If you are unsure whether your work falls
within scope for PLDI, please check the
\href{http://conf.researchr.org/track/pldi2015/pldi2015-papers#Call-for-Papers}{call
  for papers} and if still in doubt, check with the
\href{mailto:steve.blackburn@anu.edu.au?subject=[PLDI'15]}{program
  chair}.

\subsection{Concurrent Submissions and Workshops}

By submitting a manuscript to PLDI'15, authors guarantee that they
are adhering to the
\href{http://www.sigplan.org/Resources/Policies/Republication/}{SIGPLAN
  Republication Policy}. Please ensure that you are familiar with it.
Violation of any of these conditions will lead to rejection.
As always, if you are in doubt, it is best to contact the
\href{mailto:steve.blackburn@anu.edu.au?subject=[PLDI'15]}{program
  chair}.

Finally, we also note that the
\href{http://www.acm.org/publications/policies/plagiarism_policy}{ACM
  Plagiarism Policy} covers a range of ethical issues concerning the
misrepresentation of other works or one's own work.

\section{Early Access in the Digital Library}

The PLDI'15 proceedings may be available on the ACM Digital Library as
early as May 30, 2015. \textbf{Authors must consider any implications
  of this early disclosure of their work \emph{before} submitting
  their papers.}


\section{Acknowledgements}

This document is based heavily on ones prepared for previous
conferences and we thank their program chairs; in particular, Sandhya
Dwarkadas (ASPLOS'15), Sarita Adve (ASPLOS'14), Steve Keckler
(ISCA'14), Christos Kozyrakis (Micro'13), Margaret Martonosi
(ISCA'13), Onur Mutlu (Micro'12), and Michael L. Scott (ASPLOS'12).


\bibliographystyle{abbrvnat}

% We recommend that you use BibTeX.  The inlined bibitems below are
% used to keep this template to a single file.
\begin{thebibliography}{}
\softraggedright

\bibitem[Backus et~al.(1960)]{Backus:60}
J.~W. Backus, F.~L. Bauer, J.~Green, C.~Katz, J.~McCarthy, A.~J. Perlis,
  H.~Rutishauser, K.~Samelson, B.~Vauquois, J.~H. Wegstein, A.~van Wijngaarden,
  and M.~Woodger.
\newblock Report on the algorithmic language ALGOL 60.
\newblock \emph{Commun. ACM}, 3\penalty0 (5):\penalty0 299--314, May 1960.
\newblock ISSN 0001-0782.
\newblock \doi{10.1145/367236.367262}.

\bibitem[Collins(1960)]{Collins:60}
G.~E. Collins.
\newblock A method for overlapping and erasure of lists.
\newblock \emph{Commun. ACM}, 3\penalty0 (12):\penalty0 655--657, December
  1960.
\newblock \doi{10.1145/367487.367501}

\bibitem[Lamport(1994)]{Lamport:94}
L.~Lamport.
\newblock \emph{{\LaTeX: A Document Preparation System}}.
\newblock Addison-Wesley, Reading, Massachusetts, 2nd edition, 1994.

\bibitem[McCarthy(1960)]{McCarthy:60}
J.~McCarthy.
\newblock Recursive functions of symbolic expressions and their computation by
  machine, part {I}.
\newblock \emph{Commun. ACM}, 3\penalty0 (4):\penalty0 184--195, April 1960.
\newblock \doi{10.1145/367177.367199}

\end{thebibliography}

\end{document}
