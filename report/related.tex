\section{Related Work}
\label{sec:related}

We next provide an overview of related work on program repair and other algorithms
that share similar insights, ideas and techniques to our approach.

\paragraph{{\bf Code Repair.}} 
Debugging and locating errors in the code \cite{Pavlinovic:2014, Chandra:2011:AD} play 
an important role in the process of increasing software reliability. Once located, 
some errors can be easily fixed. Recently we have witnessed to a number of tools that 
aim to repair parts of code. MintHint \citep{MintHint} is such a tool that takes a more complete-program approach to code repair. Where we are using user input to guide synthesis of a correctly-typed expression from an incorrectly-typed one, MintHint targets a particular passage that is suspected to carry a bug. MintHint synthesizes some hints in areas that the algorithm considers erroneous, by both symbolically and actually executing the code and comparing its output to test cases.

\paragraph{{\bf Automated inference of program fixes and contracts.}}
The goal of the AutoFix project \cite{WeiETAL11InferringBetterContracts,PeiETAL11CodebasedAutomatedProgramFixing} is to fix
 object-oriented programs according to their
contracts. One commonality with \ourTool is the common goal of inferring code. 
In \cite{WeiETAL11InferringBetterContracts,PeiETAL11CodebasedAutomatedProgramFixing} program fixes and contracts are generated based on predefined code templates. Rather than filling 
the predefined code templates,
our approach employs smart searches through the space of all expressions that
satisfy the constraint for a successful repair.
\ourTool derives code and repairs based only on type information, without characterization of program run-time
behavior.

\paragraph{{\bf Syntactic Error Diagnosis and Repair}}
A related line of work considers a syntactic error diagnosis, as well as a recovery and a repair after such errors are found \cite{Hammond198451,Degano:1998:LTH:2245729.2245835}.
The main target of these schemes is the parsing phase within the compilation process. 
Although our algorithm focuses on the repair of expressions that are correctly parsed
according to the appropriate language grammar rules, some similarities may be noted.
With respect to these techniques and introduced taxonomy, the \ourTool algorithm may be viewed
as a type-checking repair scheme that is global, since it considers all the declarations
in the scope for the repair, and interactive, since the developer can choose between multiple
offered repair expressions and impact the repair process of subsequent errors. 
% \ivan{This sentence has more impact if our technique does *not* require a desired type,
% but rather can potentially offer multiple ones with resulting expressions (e.g.
% ??(arg1, arg2, \ldots))}
One of the most representative scheme for dealing with syntactic errors in LR and LL
parsing is presented in \cite{Burke87apractical}.
Rather than just trying combinations of insertions, substitutions and deletions
on a predefined length of the input after an error is detected, our approach
considers the abstract syntax tree of the given expression and tries to encode all valid combinations of insertions in order to produce one
or multiple expressions that would successfully type-check.


\paragraph{{\bf Searching for Better Type-error Messages}}
An interesting work, related to code repairing, has been done on improving type
errors and corresponding error messages
\cite{McAdam01repairingtype,Lerner:2006:SSM:1159876.1159887,Lerner:2007:STM:1250734.1250783}.
Although these approaches use techniques
to modify type information \cite{McAdam01repairingtype}, as well as the program
\cite{Lerner:2007:STM:1250734.1250783}, they target the problem of
unintuitive type-checker's error messages for the purpose of giving better feedback to the developer.
Our approach focuses on somewhat more ambitious goal of code repair and it does
this by preserving the type-level information as well as the structure of
ill-typed expressions.
Some techniques used in these approaches are similar to the way backbone
expressions are mutated in our approach, like argument addition and reordering
\cite{Lerner:2007:STM:1250734.1250783}.
The crucial difference is that those techniques are utilized only after an
independent search mechanism determines places for such modifications to be
done, while our approach finds all suitable repair expressions according
to the whole backbone expression and the weight heuristic.
%  and reconstructs certain number of candidate
% solutions according to the weight metric.

\paragraph{{\bf Frameworks for Deductive Synthesis and Execution of
Specifications.}} Frameworks that encompass verification, deductive
synthesis and run-time constraint solving were presented in
\cite{KuncakETAL13ExecutingSpecificationsSynthesisConstraintSolvingInvitedTalk,
KneussETAL13SynthesisModuloRecursiveFunctions}. These frameworks leverage modern SMT solvers
and use various techniques for
synthesis and verification which are integrated into an interactive tool with
 the aim to introduce specifications into the development process of reliable
 software.
% They enable the developers to compile, synthesize, execute, optimize,
% and verify mixed code fragments containing both implementation and
% specifications that are written in the common functional programming language.
% To achieve this they decompose the code into simpler fragments using predefined
% rules in a cost-driven framework, while fragments may be synthesized and
% verified at compile-time or executed at run-time by invoking a constraint
% solver.
Although these frameworks address more general goal of integrating software construction
and verification while automating multiple aspects of the development process,
in contrast to our approach that generates code fragments that do not need
to satisfy formal specification, an interesting parallel can be drawn between
such frameworks and our approach for code repair.
When given a program that contains pieces of incomplete implementation, the
aforementioned frameworks may employ techniques that synthesize missing
fragments or allow solving them at run-time.
Therefore, these frameworks can be viewed as repairing partial implementations
that may happen at compile-time or may be deferred to run-time.
The domain of both  frameworks includes only purely function programs, it is additionally
constrained by the expressiveness of the underlying SMT solver theories, verification
techniques, and deductive synthesis rules. \ourTool does not need to address those constraints, however, having only a type constraints results in a simpler code fragments that can be synthesized and repaired.

\paragraph{{\bf Sketching.}}
Program sketching tries to infer code fragments from the specification
given as separate incomplete programs
\cite{Solar-Lezama:2007:SS:1250734.1250754,Solar-Lezama:2006:CSF:1168919.1168907}.
Practicality is achieved by focusing on particular domains that allow algorithms
that employ a guided search over the syntax trees of the synthesized program with an a priori
determined bound on the syntax tree size.
While sketching requires code contracts and a program skeleton, our repair tool
 needs merely a hint of the resulting
expression and uses graph-based techniques to explore the unbounded space of expressions
that are type-correct and conform to the structure of the backbone expression.
Additionally, our approach is driven by externally defined
heuristic measures to guide the search towards better solution candidates.

\paragraph{{\bf Program Synthesis.}}
\ourTool is a repair tool that can is also used for software synthesis, since its repair algorithm is 
built on the top of a synthesis algorithm. In addition to standard synthesis tools that we use through the paper as a comparison to \ourTool, there is a number of synthesis tool with a similar functionality. CodeHint \cite{CodeHint2014ICSE} is a new dynamic approach to software synthesis 
that combines debugging and dynamic execution. CodeHint can also be used in 
interactive settings, but the main focus of CodeHint is code generation and not code repair.
XSnippet \cite{Tansalarak06xsnippet:XSnippet} and SNIFF \cite{Chatterjee:2009:Sniff} are  tools that maintain a repository of code snippets and help programmers to find suitable code. Contrary to this approach, our synthesis algorithm is dynamic as the suggested code snippets are dynamically changing with the program that the user has written up to the point when she invokes our tool.

\paragraph{{\bf Type Inference with Automatic Insertion of Coercions.}}
Finally, since coercion functions are the basis of our approach to subtyping, let us compare to the related work in that field. Several approaches on using coercions within type checking and type inference
were presented in \cite{Mitchell:1984:CTI:800017.800529,Luo:2008:CPT:1394781.1394785}.
This line of work focuses on type inference with automatic insertion of coercions in the
context of functional programming languages.
A recent work \cite{Traytel:2011:EHT:2183641.2183654} presents a more sophisticated technique that extends Hindley-Milner
type inference with coercive structural subtyping and goes beyond inserting coercions for
local expression repairs in the type inference algorithm.
The work presents an algorithm that derives subtype constraints from the whole target
term, solves them to get a substitution consistent with the partial order on types, and
finally applies the substitution to obtain a term that type-checks.
Our approach conceptually differs from the aforementioned in multiple dimensions of the
repair.
The main difference is that our approach searches for any possible expression that is
consistent with the types and structure as defined by the backbone expression.
%Additionally, our approach is limited to lambda calculus terms in the
%applicative long normal form and considers only base types without type
%constructors.
% , while the aforementioned approach
% is based on simply-typed lambda calculus with Hindley-Milner polymorphism and
% adds coercive structural subtyping.
The setup for the aforementioned approach requires mappings that define
coercions between types, while our approach leverages automatic search to find
consistent function applications.
While both approaches insert coercions only to function arguments, our approach
considers all combinations of term applications that may produce coercion terms with
appropriate types, together with any ``boarding'' arguments and mutations that may occur.




