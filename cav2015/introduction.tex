\section{Introduction}
\label{sec:intro}
While coding, a developer often knows the approximate structure of the expression she is working on, but may yet write code that does not compile because some fragments are not well-typed. Such mistakes occur mainly because modern libraries often evolve into complex application programming interfaces (APIs) that provide a large number of declarations. It is difficult, if not impossible, to learn the specifics of every declaration and its utilization.

In this paper we propose an approach that takes ill-typed expressions and automatically suggests several well-typed corrections. The suggested code snippets follow the structure outlined in the original expression as closely as possible, and are ranked based on their similarity to the original code. This approach can also be seen as code synthesis. In fact, our proposed method extends the synthesis functionality described in \cite{MandelinetALL2005Jungloid, GveroETAL13CompleteCompletionTypesWeights, PerelmanGBG12}. In light of program repair, plain expression synthesis can be seen as a repair of the empty expression.

We have implemented an early prototype of our algorithm, and empirically tested it on synthesis and repair benchmarks. The initial evaluation strongly supports the idea of a graph-based type-directed approach to code repair and snippet synthesis. Compared to the results reported in \cite{GveroETAL13CompleteCompletionTypesWeights}, our approach outperforms on similar benchmarks, sometimes by several orders of magnitude, while still producing high-quality results.
