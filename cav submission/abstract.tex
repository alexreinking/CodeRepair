Developing enterprise software often requires composing several
libraries together with a large body of in-house code. Large APIs
introduce a steep learning curve for new developers as a result of
their complex object-oriented underpinnings. While the written code in
general reflects a programmer's intent, due to evolutions in an API,
code can often become ill-typed, yet still syntactically-correct. Such
code fragments will no longer compile, and will need to be updated.

While compiler error messages provide helpful information to the
programmer regarding the location of such errors, they provide little
in the way of suggesting corrections to those errors. We describe an
algorithm suitable for integration into a compiler that automatically
repairs code expressions based on the provided almost correct
code. These suggestions would then be presented to the
programmer. Using a novel graph-theoretic approach, we efficiently
solve instances of this repair problem to synthesize a correct
expression from the salvageable parts of a broken one.
