\section{Conclusions and Future Directions}
\label{sec:concl}
We have seen that our algorithm gracefully and efficiently subsumes the work done in \cite{GveroETAL13CompleteCompletionTypesWeights, MandelinetALL2005Jungloid, PerelmanGBG12} and extends it to the problem of program repair. Using our novel graph-theoretic approach, we efficiently solve instances of this problem to synthesize a correct expression from the salvageable parts of a broken one. We believe that the algorithm in its current state has two compelling uses. First, it can assist programmers in writing complex expressions. Second, it could be integrated into a compiler to provide enhanced error messages that not only point to errors, but offer ways to correct them. Since our algorithm is sufficiently fast to run in an interactive setting, it would be appropriate to integrate some or all of its functionality into a modern IDE.
