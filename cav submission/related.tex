\section{Related Work}
\label{sec:related}

Our work is largly inspired by two synthesis tools: Prospector \cite{MandelinetALL2005Jungloid} and InSynth \cite{GveroETAL13CompleteCompletionTypesWeights, DBLP:conf/cav/GveroKP11}.
Prospector is a tool for synthesizing code snippets containing only unary API methods. The basic synthesis algorithm used in \cite{MandelinetALL2005Jungloid} encodes method signatures using a graph. We do that as well, but as explained in Sec.~\ref{sec:algorithm:graph}
our synthesis graph is more general. In nodes we do not store only types, but also descriptions how a value of that type can be constructed. In a way, those nodes correspond to succinct types introduced in  \cite{GveroETAL13CompleteCompletionTypesWeights}. Similarly, as in \cite{GveroETAL13CompleteCompletionTypesWeights}, we also consider user defined values. One can see our approach as a generalization of both those tools. Note however, that we also extend their functionality: none of the aforementioned tools supports code repair.





Debugging and locating errors in the code \cite{Pavlinovic:2014,
  Chandra:2011:AD} play an important role in the process of increasing
software reliability. Once located, some errors can be easily
fixed. Recently we have witnessed to a number of tools that aim to
repair parts of code. MintHint \cite{MintHint} is such a tool that
takes a more complete-program approach to code repair. Where we are
using user input to guide synthesis of a correctly-typed expression
from an incorrectly-typed one, MintHint targets a particular passage
that is suspected to carry a bug. MintHint synthesizes some hints in
areas that the algorithm considers erroneous, by both symbolically and
actually executing the code and comparing its output to test cases.



