\section{Motivating Examples}
\label{sec:examples}
We first illustrate how our algorithm works and the nature of repair with the usage of familiar API calls from the Java standard library, reflecting along the way on the differences between our and other tools for code completion and repair. To emphasize the significance in practical software development, our examples are chosen from a set of the real code examples (most of them featured at \url{http://www.java2s.com/}{}).

\subsection{Code Correction: Single Error}
\label{sec:examples:single}
The following simple example, adapted from \cite{GveroETAL13CompleteCompletionTypesWeights}, illustrates how our algorithm handles an expression that does not compile. Consider the following program fragment:
\begin{lstlisting}
public class Main {
  public static void main(String args[]) throws IOException {
    String body = "email.txt"; String sig = "sign.txt";
    SequenceInputStream seqStream = new SequenceInputStream(body, sig); // error
  }
} 
\end{lstlisting}
Here, the developer has given the \lstinline{SequenceInputStream} constructor two strings, even though none of its overloads accept those arguments. Still, we can recognize the intent of the intended expression. To preserve this intent, our algorithm should construct an expression that maintains the relative position of the original arguments. The corrected expression should favor reusing the well-typed subexpressions, such as the \lstinline{body} and \lstinline{sig} variables in their corresponding sub-expression trees. Our algorithm finds all such expressions, constructed from the visible declarations in the current scope. The results are well-typed and ranked according to a metric that favors resemblance to the starting expression.

In this case, the returned expression with the lowest cost is \lstinline{new SequenceInputStream(new FileInputStream(body), new FileInputStream(sig))}, which precisely reflects the programmer's intent.

\subsection{Code Correction: Many Errors}
\label{sec:examples:many}
Sometimes, an ill-typed expression poorly reflects the structure of the desired expression, while still retaining other useful information. Often, this is a result of passing too many or too few arguments arguments to a function, or passing them in the wrong order.

The following code uses an extensive number of calls to the standard Java API library with the intention to read a compressed file though a buffered stream. To read the file, the developer attempts to instantiate an \lstinline{InputStream} object:
\begin{lstlisting}
import java.io.*;
import java.util.zip.*;

public class Main {
  public static void main(String args[]) throws IOException {
    int buffSize = 1024, compLevel = Deflater.BEST_SPEED;
    String fileName = "compressed.txt";
    
    InputStream input = 
      new BufferedInputStream(buffSize, new DeflaterInputStream(
        new FileInputStream(), compLevel, true)); // error 
  }
}
\end{lstlisting}
In this example, the single expression contains three errors. First, the \lstinline{FileInputStream} constructor requires at least one parameter; second, the \lstinline{DeflaterInputStream} constructor has been passed too many arguments; and finally, the \lstinline{BufferedInputStream} has been passed valid arguments in the wrong order.

To repair this expression, our algorithm proceeds from the bottom of the expression, viewed as a parse tree, up to the top-level expression. The repair algorithm begins with the sub-expression \lstinline{new FileInputStream()}, and returns \lstinline{new FileInputStream(fileName)} as the top expression. After applying this sub-repair, the algorithm proceeds to correct the \lstinline{DeflaterInputStream}. Since all of its arguments are now well-typed, it adds each one to a list to indicate they should be reused. The evaluation of the cost of an expression cross-references this list to scale down the cost of the snippets containing those expressions. The top repair is found by wrapping the extra arguments in the \lstinline{Deflater} constructor. At this stage, we have already built the following correct subexpression:
\begin{lstlisting}
new DeflaterInputStream(new FileInputStream(fileName), new Deflater(compLevel, true))
\end{lstlisting}
Finally, the algorithm proceeds to rebuild the overall expression by interchanging the arguments in the top-level, almost-repaired expression to arrive at our final, correct result:
\begin{lstlisting}   
new BufferedInputStream(new DeflaterInputStream(
  new FileInputStream(fileName), new Deflater(compLevel, true)),
    buffSize);
\end{lstlisting}
These examples show that our algorithm can efficiently search through a large number of possible repair expressions, and can guide that procedure according to an appropriate metric that values shorter distances between the given expression and the corrected expression.

We believe that our algorithm will perform useful and effective repairs that are well-aligned with the developer's intentions, even when the given ill-typed expression requires several steps to repair.
